\documentclass[]{article}
\usepackage{graphicx}
\usepackage{amsmath}

%opening
\title{Homework \#4}
\author{Claire Jung}

\begin{document}

\maketitle
Collaborated with Natasha Antropova
\section*{Problem 1}
We need to show that the two training objectives are equivalent. Let's call,
\[
A = \sum_{i=1}^{N}(y_i-(w\cdot \phi(x_i)+w_0))^2 + \lambda||w||^2
\]
\[
B =  \sum_{i=1}^{N}(y_i-(w\cdot \phi(x_i)+w_0))^2 + \alpha(||w||^2-\tau)
\]
Where we have used a Lagrange multiplier, $\alpha$, for the constraint.

For the optimal solutions, we know:
\[
\nabla_wA(w^*_\lambda)=0
\]
and using the KKT conditions,
\[
\nabla_wB(w^*_\alpha)=\nabla_wA(w^*_\alpha)=0 \textrm{ and } \alpha(||w^*_\alpha||^2 - \tau) = 0
\]
where $w^*_\lambda$ are the optimal weights for a given $\lambda$ and $w^*_\alpha$ are the optimal weights for a given $\alpha$.

Say we set $\tau=||w^*_\lambda||^2$, then we have $\lambda = \alpha$ and $w^*_\lambda = w^*_\alpha$, satisfying the KKT condition. 

Therefore, the optimal weights will be the same for a $\tau=||w^*_\lambda||^2$.
\section*{Problem 2}
See attached notebook.

\section*{Problem 3}
The posterior probability $\gamma_{i,c} = p(z_i=c|x_i;\theta,\pi)$ may be written as,
\[
\gamma_{i,c} = \frac{\pi_c p(x_i|\theta_c)}{\sum_{l=1}^{k}\pi_l p(x_i|\theta_l)}
\]
For a Bernoulli mixture,
\[
p(x_i|\theta_c) = \prod_{j=1}^{d}\theta_{c,j}^{x_{i,j}}(1-\theta_{c,j})^{(1-x_{i,j})}
\]
Therefore,
\[
\gamma_{i,c} = \frac{\pi_c \prod_{j=1}^{d}\theta_{c,j}^{x_{i,j}}(1-\theta_{c,j})^{(1-x_{i,j})}}{\sum_{l=1}^{k}\pi_l \prod_{j=1}^{d}\theta_{l,j}^{x_{i,j}}(1-\theta_{l,j})^{(1-x_{i,j})}}
\]
\section*{Problem 4}
The M-step updates are,
\[
\pi^{new},\theta^{new} = argmax_{\pi,\theta}\big[\sum_{i=1}^{N}\sum_{c=1}^{k}\gamma_{i,c}(\log\pi_c + \log p(x_i;\theta_c))\big] 
\]
Let us set,
\[
A = \sum_{i=1}^{N}\sum_{c=1}^{k}\gamma_{i,c}(\log\pi_c + \log p(x_i;\theta_c)) = \sum_{i=1}^{N}\sum_{c=1}^{k}\gamma_{i,c}(\log\pi_c + \sum_{j=1}^{d}[x_{i,j}\log\theta_{c,j}+(1-x_{i,j})\log(1-\theta_{c,j})])
\]
Solving for $\pi^{new}$ using Lagrange multipliers and differentiating,
\[
A = \sum_{i=1}^{N}\sum_{c=1}^{k}\gamma_{i,c}(\log\pi_c)+\alpha(\sum_{c=1}^{k}\pi_c -1) + ...
\]
\[
\frac{\partial A}{\partial \pi_c} = 0 \rightarrow \sum_{i=1}^{N}\gamma_{i,c} = -\alpha\pi_c
\]
\[
\sum_{c=1}^{k}\sum_{i=1}^{N}\gamma_{i,c} = -\sum_{c=1}^{k}\alpha\pi_c\rightarrow\sum_{i=1}^{N}1 = -\alpha \rightarrow \alpha = -N
\]
\[
\pi_k^{new} = \frac{1}{N}\sum_{i=1}^{N}\gamma_{i,k}
\]
for a given subclass k.

Solving for $\theta^{new}$,
\[
\frac{\partial A}{\partial \theta_{k,r}} =0= \sum_{i = 1}^{N}\gamma_{i,k}\Big(\frac{x_{i,r}}{\theta_{k,r}}-\frac{1-x_{i,r}}{1-\theta_{k,r}}\Big) = \sum_{i = 1}^{N}\gamma_{i,k}(x_{i,r}-\theta_{k,r})  = 0
\]
\[
\theta^{new}_{k,r} = \frac{\sum_{i=1}^{N}\gamma_{i,k}x_{i,r}}{\sum_{i=1}^{N}\gamma_{i,k}}
\]
for a given subclass k and parameter r.
\section*{Problem 5}
See attached notebook.
\end{document}